\usepackage{capt-of}
\usepackage{tcolorbox}
\usepackage{varwidth}
\usepackage{amsmath}

\tcbuselibrary{breakable}
\tcbuselibrary{skins}
\tcbuselibrary{xparse}%xparse . sty定义的NewDocumentCommand
\definecolor{deficolor}{RGB}{200,44,44}
\definecolor{theocolor}{RGB}{0,0,0}
\definecolor{praccolor}{RGB}{49,44,120}

\newcounter{theorem}
\numberwithin{theorem}{section}% numberwithin是amsmath定义
\NewTColorBox{theobox}{o m +o m}{% o省略可能的参数、m必备的参数、+是多行里,也可以表示。
%# 1 =标题(省略可), # 2 =定理环境名,# 3 = tcolorbox追加的选项
enhanced,frame empty,interior empty,
coltitle=white,fonttitle=\bfseries,colbacktitle={#4},
extras broken={frame empty,interior empty},
borderline={0.5mm}{0mm}{#4},
sharp corners=downhill,
breakable=true,
top=4mm,
before skip=3.5mm,
attach boxed title to top left={yshift=-3mm,xshift=5mm},
boxed title style={boxrule=0pt,sharp corners=all},varwidth boxed title,
IfNoValueTF={#1}{title=#2~\thetheorem.}{title=#2~\thetheorem:~#1},
IfNoValueTF={#3}{}{#3}%第三个参数省略- NoValue -返回
}

\NewDocumentEnvironment{theorem}{o m +o m}{%NewDocumentEnvironment由xparse.sty定义
\refstepcounter{theorem}\begin{theobox}[#1]{#2}[#3]{#4}}{\end{theobox}}

\NewDocumentEnvironment{defi}{o +d""}{%d""表示作为省略的一部分。
\begin{theorem}[#1]{定义}[#2]{{deficolor}}
}{\end{theorem}}

\NewDocumentEnvironment{theo}{o +d""}{%d""表示作为省略的一部分。
\begin{theorem}[#1]{定理}[#2]{{theocolor}}
}{\end{theorem}}

\NewDocumentEnvironment{prac}{o +d""}{%
\begin{theorem}[#1]{练习}[#2]{{praccolor}}
}{\end{theorem}}